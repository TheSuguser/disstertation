\maketitle
\makedeclaration

\begin{abstract} % 300 word limit
This dissertation investigates the use of concept drift framework and deep learning techniques in fashion news popularity prediction. A new design of a deep neural work in text classification is also presented. The dissertation consists of two key experiments:

\subsection*{Experiment 1: Concept Drift Detection}
The first experiment applies the concept drift detection algorithm to predict the popularity of the fashion news. More specifically, it investigates the performance between the concept drift detection framework and traditional incremental learning framework. Methodologies, such as drift detection method(DDM), Multinomial Bayes classifier, are proposed to classify the fashion news. The dataset for this experiment, collected by the web crawler, is the four-year fashion news from www.hypebeast.com. It contains 87,517 fashion news labeled by categories, keywords which including the related celebrities and brands, and the index of popularity. 

\subsection*{Experiment 2: Multi-Input Deep Neural Network in Fashion News Popularity Prediction}
The second experiment proposes a new structure of deep neural network in text classification. The new model combines the advantages of fastText model and bidirectional long-term short memory model and resorts the techniques commonly used on images, such as focal loss, and spatial dropout. The effectiveness of the proposed model in improving the prediction performance is evaluated by comparing with four state-of-the-art deep learning models, including fastText, Character-level CNN, bidirectional LSTM, and attention based bidirectional LSTM. The proposed model is trained on the fashion news from 1/1/2014 to 31/3/2018 and evaluated on the news from 1/4/2018 to 13/6/2018.

\section*{Contributions to Science}
The major contribution of this dissertation is to offer a text-based deep learning model to predict the popularity of the fashion news. The prediction results can be used to 1) solve the cold start problem of the news providers' recommendation system; 2) Track the fashion trend quantitatively; 3) indicate the market attention to the fashion retail companies.

This thesis contributes to the existing literature in a number of ways. First, this research investigates the application of the concept drift detection algorithm in real world data, especially the high-dimensional and noisy text data. Most published concept drift detection algorithms are evaluated in the low-dimensional and noise-free dataset and few of them were applied to text data. Second, the research proposed a new structure of the deep learning model in text prediction problem which could achieve a competitive results. Third, the experiment proves that the techniques commonly used on image data also have a wide application on text data. 



\end{abstract}

\begin{acknowledgements}
Acknowledge all the things!
\end{acknowledgements}

\setcounter{tocdepth}{2} 
% Setting this higher means you get contents entries for
%  more minor section headers.

\tableofcontents
\listoffigures
\listoftables

