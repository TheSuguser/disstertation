\chapter{Introduction}
\label{introduction}

\begin{quote}
This chapter presents an overview of the dissertation. First, the motivation for this research is introduced along with the research topic. Second, the objectives of this research are discussed. Third, this chapter described the research methodology. This chapter ends by describing the structure of the whole dissertation.
\end{quote}

\section{Motivation}
Predicting fashion news popularity from on-line fashion media is critically important from many aspects. First, there are many different news agencies on-line and they posted dozens of news every day. It is impossible to read all of them. Many website already have the recommendation system of popular news based on the page-views. As they use the page-views which have already occurred , it is not timely and has a cold start problem. The prediction of the popularity can highly avoid the cold start problem and provide the important features to the recommendation system. Second, it benefits the advertisement. A popular news means it will have more page-views in the future and namely has a higher value for advertisement. In this case, an effective fashion news popularity prediction enables the fashion medias to maximize revenue through differential pricing for access to contents with different popularity. Third, fashion trend is a very subjective concept and is hard to be described quantitatively. By predicting the popularity, it is possible to observe the change of fashion objectively and obtain a quantitative indicator which reflect the fashion hot-spot. Finally, some recent research also suggests that the very early indicators can be extracted from news to predict changes in various economic and commercial indicators. The popularity of the fashion news is, to a certain extent, an indicator of market attention, which can be helpful feature in stock price movement prediction.

Many research tried to track the popularity of the news and tweets based on the information propagation model, social dynamic model and the virus spreading model rather than news themselves. Wu and Shen ~\cite{wu2015analyzing} developed a model which can predict a news tweet’s popularity based on the number of its retweets soon after being published. Lerman and Hogg~\cite{lerman2010using} used early user reactions to new content to create the stochastic models of user behavior on the website to predict the popularity.Rizos and el.\cite{rizos2016predicting} predicted the news story popularity indicators based on the user graphs. Unlike these work that paid little attention on the news itself, we particularly study the popularity of news from the text content.

Popularity prediction is a very timely task as we can obtain the ground truth of the popularity after a while and the prediction result is not needed anymore. It requires the model to not only accurate but timely. Some of the models aforementioned have the cold start problem. Because there is no information flow when the news is just released. However, there is not such problem for a context-based model. In addition, dozen of news are posted everyday, the model must be updated everyday. Most of the published model use SVM, Random forest. It is not easy to do the incremental learning with these algorithms. It means the model can only updated by training with th whole data set which is very inefficient. However, deep learning model can easily support the incremental training and update the model quickly with the new instance. 

The first challenge is that fashion hot-spot is not stable. From a machine learning perspective, fashion trend may depend on some hidden context, not given explicitly in the form of predictive features. In other words, tracking fashion trend from fashion news is a concept drift problem. Thus, the conventional prediction framework may not work in such problem. It is necessary to create a system to detect the concept drift phenomenon and exam the performances between the concept-drift model and conventional machine learning model. 

The second challenge is that fashion trend can change both gradually and suddenly~\cite{behling1985}. Namely, the underlying data distribution may change gradually or suddenly with time. Unfortunately, none of published algorithm has a good performance in handling both change. Some algorithms may overreact to the noise, erroneously interpreting it as concept drift, while others may be highly robust to noise, adjusting to the changes too slowly~\cite{widmer1996learning}. In addition, most of the published algorithms were only tested with the low-dimensional concept drift problem, while in the most cases, text classification problem is high-dimensional. 

In addition to the first two challenges aforementioned, the third challenge is that the data set is imbalanced. Most machine learning algorithms works best when the number of instances of each class are roughly equal. In this research, popular news only accounts for less than 10\% of all the news. Imbalance data set may leads to accuracy paradox. In this case, where the accuracy measures tell the story that the model has excellent accuracy, but the accuracy is only reflecting the underlying class distribution. Thus, it is challenging to select a suitable model and metrics. It is worth trying a variety of techniques to balance the dataset. 

Finally, there are many published text classification model based on deep learning right now. Each of them has very different structure and mechanism, thus has its own pros and cons. It is necessary to do multiple experiments to select the effective structure and layers. The new model can be developed based on the results of these experiments

The main motivation of this research is to improve the accuracy of the fashion news popularity prediction model to help fashion media improve their recommendation system, advertising business and provide a quantitative indicator for fashion trend. 


\section{Objective}
Our research objective is to improve the accuracy of predicting the popularity of the fashion news by testing the published concept drift algorithm and propose a new deep learning model. The main hypothesis of this research states that:

\begin{quote}
The fashion trend can be tracked by analyzing the popularity of the fashion news. Current published concept drift algorithm do not have obvious advantage compared with the conventional machine learning framework. By exam the state-of-the-art deep classification model, the new model can significantly increase the accuracy of the fashion news popularity prediction.

\end{quote}

To validate this hypothesis, there are two main tasks for this research:

\subsection*{1. Build a concept drift detection framework.}
The first task is to detect the concept drifts and compare the performances between the framework which can handling the concept drift problem and the classical machine learning framework. It can be done easily by comparing the metrics of prediction result of the test data. This research, therefore, gives the conclusion if the concept drift increase the performance of the prediction model. 



\subsection*{2. Propose a system design for predicting the fashion news popularity}
Based on the first experiment's result, the second task is to construct a model to predict the fashion new popularity. First, four state-of-the-art text classification models are implemented and their performances are compared.
Then, based on the comparison, the new model structure is proposed. This research presents a multiple inputs and outputs model which can deal with the cold start problem and easily be updated. 


\section{Methodology}
\subsection{Data used in this research}
The dataset for this dissertation, crawled from the famous fashion websites http://www.hypebeast.com and http://www.hypebae.com, contains 12-years fashion news labeled with keywords, categories, comments, and index of HYPE. The official definition of HYPE is 

\begin{quote}
HYPES is a real time popularity metric informing reader of what is trending
\end{quote}

It contains 182,294 fashion news from 2005-4-20 to 2018-6-14. According to the experience, the website also divides the index of HYPES into three classes shown in Table \ref{popularityclasses}.

\begin{table}[]
\centering
\caption{Popularity Classes}
\label{popularityclasses}
\begin{tabular}{ll}
\hline
Popularity & HYPES               \\ \hline
Hot        & \textgreater{}20000 \\
Medium     & 5000-20000          \\
Cold       & 0-5000 \\
\hline             
\end{tabular}
\end{table}

Hypebeast.com is the most famous fashion and street culture website around world, attracting 9.4 million unique visitor per month and 44 million page views per month. Its social media account has 9.3 million followers. It can be said that HYPEBEAST is the most influential fashion and street culture media.


\subsection{Model design}
In this research, there are two models which need to be designed. The first one is the model that uses the concept drift framework. The second model is the deep learning model combines a variety of the state-of-the-art neural network layers.

In the part of the first model, the Drift Detection Method (DDM) is applied to detect the concept drift. In this method we assume that examples arrive one at a time. The framework could be easy extended to situations where data comes on batches of examples. We consider the online learning framework. In this framework when an example becomes available, the decision model must take a decision (e.g. an action). Only after the decision has been taken the environment react providing feedback to the decision model (e.g. the class label of the example)~\cite{gama2004learning}.

The second model is the deep learning model with multiple inputs and outputs. There are three inputs layers. The first one receive the tokenized titles and embed them. The embedded titles are feed into the average pooling layer to extract the features. The tokenized articles are sent to the second input layer to be embedded. The two layer bidirectional GRU are used to deal with the sequence input and the extracted features are concatenated with the feature of title. The concatenated feature vector is connect to a dense layer and output the prediction result of the cold start fashion news. For a non cold start problem, there is another input to receive the comments from the readers. After feature engineering, the feature vector of the comments are then concatenated with the other two vectors and connected to the dense layers. The prediction result for the non cold start news is produced by these dense layers.

\subsection{Model implementation}
All the models are implemented in Python. The deep learning model is implemented by Tensorflow and Keras. The raw data is collected by the website crawler implemented by Scrapy library. In addition, MongoDB is used as the database to store the raw data. 



\section{Structure of the dissertation}
Structure of this dissertation is as follows,

\begin{itemize}
	\item Chapter 2 reviews background information on a number of key concepts in the area of Machine Learning, concept drift problem, text classification problem. It also presents the basic theory of deep learning and the state-of-art neural network structure in text classification. The literature related to the research is also briefly described in this chapter.

	\item Chapter 3 describes the state-of-the-art framework for concept drift detection and compare the performance of the concept drift framework and the conventional machine learning framework. 

	\item Chapter 4 exams the performance of four published deep classification models. Based on the experiment results, a new model is proposed. The model test also shows in this chapter. The investigation of the model is presented as well.

	\item Chapter 6 summarises this research and outlines possibilities for future work.


\end{itemize}

